\documentclass[12pt,a4paper]{article}\usepackage[]{graphicx}\usepackage[]{color}
%% maxwidth is the original width if it is less than linewidth
%% otherwise use linewidth (to make sure the graphics do not exceed the margin)
\makeatletter
\def\maxwidth{ %
  \ifdim\Gin@nat@width>\linewidth
    \linewidth
  \else
    \Gin@nat@width
  \fi
}
\makeatother

\definecolor{fgcolor}{rgb}{0.345, 0.345, 0.345}
\newcommand{\hlnum}[1]{\textcolor[rgb]{0.686,0.059,0.569}{#1}}%
\newcommand{\hlstr}[1]{\textcolor[rgb]{0.192,0.494,0.8}{#1}}%
\newcommand{\hlcom}[1]{\textcolor[rgb]{0.678,0.584,0.686}{\textit{#1}}}%
\newcommand{\hlopt}[1]{\textcolor[rgb]{0,0,0}{#1}}%
\newcommand{\hlstd}[1]{\textcolor[rgb]{0.345,0.345,0.345}{#1}}%
\newcommand{\hlkwa}[1]{\textcolor[rgb]{0.161,0.373,0.58}{\textbf{#1}}}%
\newcommand{\hlkwb}[1]{\textcolor[rgb]{0.69,0.353,0.396}{#1}}%
\newcommand{\hlkwc}[1]{\textcolor[rgb]{0.333,0.667,0.333}{#1}}%
\newcommand{\hlkwd}[1]{\textcolor[rgb]{0.737,0.353,0.396}{\textbf{#1}}}%
\let\hlipl\hlkwb

\usepackage{framed}
\makeatletter
\newenvironment{kframe}{%
 \def\at@end@of@kframe{}%
 \ifinner\ifhmode%
  \def\at@end@of@kframe{\end{minipage}}%
  \begin{minipage}{\columnwidth}%
 \fi\fi%
 \def\FrameCommand##1{\hskip\@totalleftmargin \hskip-\fboxsep
 \colorbox{shadecolor}{##1}\hskip-\fboxsep
     % There is no \\@totalrightmargin, so:
     \hskip-\linewidth \hskip-\@totalleftmargin \hskip\columnwidth}%
 \MakeFramed {\advance\hsize-\width
   \@totalleftmargin\z@ \linewidth\hsize
   \@setminipage}}%
 {\par\unskip\endMakeFramed%
 \at@end@of@kframe}
\makeatother

\definecolor{shadecolor}{rgb}{.97, .97, .97}
\definecolor{messagecolor}{rgb}{0, 0, 0}
\definecolor{warningcolor}{rgb}{1, 0, 1}
\definecolor{errorcolor}{rgb}{1, 0, 0}
\newenvironment{knitrout}{}{} % an empty environment to be redefined in TeX

\usepackage{alltt}

\usepackage{amsfonts}
\usepackage{hyperref}

\author{James Foadi \\ 
        email \href{mailto:j.foadi@bath.ac.uk}{j.foadi@bath.ac.uk}}

\title{A quick introduction to crone}

%\VignetteIndexEntry{A quick introduction to crone}
%\VignetteEngine{knitr::knitr}
\IfFileExists{upquote.sty}{\usepackage{upquote}}{}
\begin{document}

\maketitle

The \texttt{crone} package has been developed to allow students in quantitative subjects coming to terms with the common operations and algorithms used in x-ray structural crystallography. This discipline deals with the determination of atomic positions and average thermal vibrations for molecules arranged in highly-ordered 3D crystalline lattices. To avoid the (mostly notational) complications of 3D mathematics and in order to allow a clearer and transparent handling of the main mathematical concepts and calculations, \texttt{crone} defines atoms, molecules and crystal structures in 1D, rather than 3D. While there are 230 types of crystal symmetry (groups) in 3D, there are only two symmetries in 1D, called $P1$ and $P\bar{1}$. A few $P1$ and $P\bar{1}$ structures are included in the package to provide data for various demonstrations. Arbitrary 1D structures can also be defined by the user at any time.

In this quick introduction two different 1D structures will be loaded from the \texttt{crone} library into the working space. Graphical representations and some crystallographic calculations will follow, with the aim of demonstrating some of the package's possible uses

\section{Data preparation}
A certain number of so-called \emph{linear molecules} are included in the package. To load data corresponding to these molecules you have to use function \texttt{load\_structure}. If the function is called without argument (default is \texttt{NULL}), then the list of all 1D structures available is returned:

\begin{knitrout}
\definecolor{shadecolor}{rgb}{0.969, 0.969, 0.969}\color{fgcolor}\begin{kframe}
\begin{alltt}
\hlkwd{require}\hlstd{(crone)}
\end{alltt}


{\ttfamily\noindent\itshape\color{messagecolor}{\#\# Loading required package: crone}}\begin{alltt}
\hlkwd{load_structure}\hlstd{()}
\end{alltt}
\begin{verbatim}
## 1D structures available for loading:
## 
##    beryllium_fluoride
##    carbon_dioxide
##    cyanate
##    nitronium
##    thiocyanate
##    xenon_difluoride
\end{verbatim}
\end{kframe}
\end{knitrout}

Let's try and load data corresponding to carbon dioxide:

\begin{knitrout}
\definecolor{shadecolor}{rgb}{0.969, 0.969, 0.969}\color{fgcolor}\begin{kframe}
\begin{alltt}
\hlcom{# Make sure to type the underscore!}
\hlstd{sdata} \hlkwb{<-} \hlkwd{load_structure}\hlstd{(}\hlstr{"carbon_dioxide"}\hlstd{)}

\hlcom{# The object returned by load_structure is a named list}
\hlkwd{class}\hlstd{(sdata)}
\end{alltt}
\begin{verbatim}
## [1] "list"
\end{verbatim}
\begin{alltt}
\hlkwd{names}\hlstd{(sdata)}
\end{alltt}
\begin{verbatim}
## [1] "a"   "SG"  "x0"  "Z"   "B"   "occ"
\end{verbatim}
\end{kframe}
\end{knitrout}

\section{Conclusions}

Elliptic functions are an interesting and instructive branch of
complex analysis, and are frequently encountered in applied
mathematics: here they were used to calculate a potential flow field
in a rectangle.



\subsection*{Acknowledgements}
I would like to acknowledge the many stimulating and helpful comments
%\bibliography{elliptic}
\end{document}
